\chapter*{Ejercicio 14 Cápitulo 7 ABD }

\textbf{14.-} Una partícula que se mueve a lo largo del eje-x, tiene una función velocidad:
\[
v(t) = t^{3}e^{-t}
\]
Que tan lejos la partícula viaja desde el tiempo $t=0$ hasta $t=5$. \\
Para encontrar lo que recorrío debemos resolver la integral de la función $v(t)$ en el intervalo dado:(D = Distancia)
\[
D = \int_{0}^{5} v(t) dt
\]
Es decir:
\[
D = \int_{0}^{5} t^{3}e^{-t}  dt
\]
Para resolver esta integral, lo haremos por Integración por Partes y haciendo uso de:
\[
\int u \, dv = uv - \int v \, du 
\]
Resolviendo:
\[
D = \int_{0}^{5} t^{3}e^{-t}  dt
\]
Con:
\begin{multicols}{2}
	\noindent
	\begin{align*}
		u &= t^{2}  \\
        du &= 2tdt \\
	\end{align*}
	\columnbreak
	\begin{align*}
	    dv &= e^{-t}  dt \\
        \int \, dv &= \int e^{-t}  \, dt \\
        v &= -e^{-t}
	\end{align*}
\end{multicols}
Sust en la Fórmula de Integración por Partes:
\begin{align}
    D = (t^{2})(-e^{-t}) - \int_{0}^{5} -e^{-t} 2t \,  dt 
\end{align}
\begin{align*}
    D = (t^{2})(-e^{-t}) + 2\int_{0}^{5} e^{-t} t \,  dt \\
\end{align*}
Ahora, para resolver:
\[
\int_{0}^{5} e^{-t} t \,  dt
\]
Usaremos:
\begin{multicols}{2}
	\noindent
	\begin{align*}
		u&= t  \\
        du &= dt \\
	\end{align*}
	\columnbreak
	\begin{align*}
	    dv &= e^{-t}  dt \\
        \int \, dv &= \int e^{-t}  \, dt \\
        v &= -e^{-t}
	\end{align*}
\end{multicols}
Así:
\begin{align*}
    \int_{0}^{5} e^{-t} t \,  dt &= (t)(-e^{-t}) - \int-e^{-t} \, dt \\
    \int_{0}^{5} e^{-t} t \,  dt &= (t)(-e^{-t}) + \int e^{-t} \, dt \\
    \int_{0}^{5} e^{-t} t \,  dt &= (t)(-e^{-t}) - e^{-t}  \\
    \int_{0}^{5} e^{-t} t \,  dt &= (e^{-t})(-t-1)  \\
    \int_{0}^{5} e^{-t} t \,  dt &= (-e^{-t})(t+1)  \\
\end{align*}
Reemplazado de vuelta en (1):
\begin{align*}
    D = (t^{2})(-e^{-t}) +2 \int_{0}^{5} e^{-t} t \,  dt &= (t^{2})(e^{-t}) +2 [(-e^{-t})(t+1)] \Bigg]_{0}^{5}\\
    D &= (t^{2})(-e^{-t}) +2 (-e^{-t})(t+1) \Bigg]_{0}^{5} \\
    D &= (-e^{-t})[t^{2} + 2(t+1)]\Bigg]_{0}^{5} \\
    D &= (-e^{-t})[t^{2} + 2t +2]\Bigg]_{0}^{5} \\
    D &= (-e^{-t})[t^{2} + 2t +2]\Bigg]_{0}^{5} \\
    D &= [(-e^{-(5)})[(5)^{2} + 2(5) +2]] - [(-e^{-(0)})[(0)^{2} + 2(0) +2]]\\
    D &= [(-e^{-(5)})[25 + 10 +2]] - [(-e^{-(0)})[0 + 0 +2]]\\
    D &= [(-e^{-(5)})[37]] - [(-e^{-(0)}[2]]\\
    D &= [(-e^{-(5)})[37]] - [-\frac{1}{e^{0}}[2]]\\
    D &= [(-e^{-(5)})[37]] - [-\frac{1}{1}[2]]\\
    D &= -37e^{-(5)} - [-2]\\
    D &= -37e^{-(5)} +2\\
\end{align*}
Así la partícula se movió:\\
\[
D = -37e^{-5} +2 \text{ unidades} 
\]
o
\[
D = - \frac{37}{e^{5}} +2 \text{ unidades} 
\]
