\chapter*{Ejercicio 42 Capitulo5}
\textbf{Calcular el área bajo la curva \( y = \frac{1}{x} \) en el intervalo \([1, e^3]\).}

El área bajo la curva está dada por la integral definida:
\[
\int_{1}^{e^3} \frac{1}{x} \, dx
\]

\subsection*{Plantear la integral}
Tenemos:
\[
\int_{1}^{e^3} \frac{1}{x} \, dx
\]

\subsection*{Calcular la integral indefinida}
Sabemos que:
\[
\int \frac{1}{x} \, dx = \ln|x| + C
\]

\subsection*{Evaluar la integral definida}
Sustituimos los límites de integración:
\[
\int_{1}^{e^3} \frac{1}{x} \, dx = \left[ \ln x \right]_{1}^{e^3}
\]

\subsection*{Sustituir los límites}
Evaluamos en los límites:
\[
\left[ \ln x \right]_{1}^{e^3} = \ln(e^3) - \ln(1)
\]

\subsection*{Simplificar}
Sabemos que:
\[
\ln(e^3) = 3 \quad \text{y} \quad \ln(1) = 0
\]
Por lo tanto:
\[
\ln(e^3) - \ln(1) = 3 - 0 = 3
\]

\subsection*{Resultado final}
El área bajo la curva es:
\[
\boxed{3}
\]
\newpage
