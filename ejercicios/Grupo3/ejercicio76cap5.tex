\chapter*{Ejercicio 76 capítulo 5 ABD}

La función de velocidad dada es:
\[
v(t) = \frac{2}{5} \sqrt{5t + 1} + \frac{8}{5}.
\]

Queremos calcular el \textbf{desplazamiento} y la \textbf{distancia} recorrida por la partícula en el intervalo \( [0, 3] \).

\subsection*{Desplazamiento}

El desplazamiento se calcula como la integral de la velocidad:
\[
\text{Desplazamiento} = \int_{0}^{3} v(t) \, dt.
\]
Sustituimos \( v(t) \) en la integral:
\[
\int_{0}^{3} v(t) \, dt = \int_{0}^{3} \left( \frac{2}{5} \sqrt{5t + 1} + \frac{8}{5} \right) \, dt.
\]
Separando la integral:
\[
\int_{0}^{3} v(t) \, dt = \frac{2}{5} \int_{0}^{3} \sqrt{5t + 1} \, dt + \frac{8}{5} \int_{0}^{3} 1 \, dt.
\]

\subsubsection*{Primera integral: \( \int_{0}^{3} \sqrt{5t + 1} \, dt \)}

Sea \( u = 5t + 1 \), por lo tanto:
\[
du = 5 \, dt \quad \text{y} \quad dt = \frac{1}{5} \, du.
\]
Cuando \( t = 0 \), \( u = 1 \); y cuando \( t = 3 \), \( u = 16 \). Sustituyendo:
\[
\int_{0}^{3} \sqrt{5t + 1} \, dt = \int_{1}^{16} \sqrt{u} \cdot \frac{1}{5} \, du = \frac{1}{5} \int_{1}^{16} u^{1/2} \, du.
\]
La integral de \( u^{1/2} \) es:
\[
\int u^{1/2} \, du = \frac{2}{3} u^{3/2}.
\]
Entonces:
\[
\frac{1}{5} \int_{1}^{16} u^{1/2} \, du = \frac{1}{5} \left[ \frac{2}{3} u^{3/2} \right]_{1}^{16} = \frac{2}{15} \left[ u^{3/2} \right]_{1}^{16}.
\]
Evaluamos los límites:
\[
\frac{2}{15} \left[ 16^{3/2} - 1^{3/2} \right] = \frac{2}{15} \left[ (16)^{3/2} - 1 \right].
\]
Sabemos que \( 16^{3/2} = (16^{1/2})^3 = 4^3 = 64 \), entonces:
\[
\frac{2}{15} \left[ 64 - 1 \right] = \frac{2}{15} \cdot 63 = \frac{126}{15} = \frac{42}{5}.
\]

\subsubsection*{Segunda integral: \( \int_{0}^{3} 1 \, dt \)}

La integral es directa:
\[
\int_{0}^{3} 1 \, dt = \left[ t \right]_{0}^{3} = 3 - 0 = 3.
\]

\subsubsection*{Combinando ambas integrales}

Sustituyendo los resultados en la expresión original:
\[
\int_{0}^{3} v(t) \, dt = \frac{2}{5} \cdot \frac{42}{5} + \frac{8}{5} \cdot 3 = \frac{84}{25} + \frac{24}{5}.
\]
Simplificamos \( \frac{24}{5} \) a denominador \( 25 \):
\[
\frac{24}{5} = \frac{120}{25}.
\]
Entonces:
\[
\int_{0}^{3} v(t) \, dt = \frac{84}{25} + \frac{120}{25} = \frac{204}{25}.
\]

Por lo tanto, el desplazamiento es:
\[\boxed{
\text{Desplazamiento} = \frac{204}{25} \, \text{m}}
\]

\subsection*{Distancia recorrida}

La distancia recorrida es la misma que el desplazamiento, ya que \( v(t) \geq 0 \) en todo el intervalo \( [0, 3] \). Entonces:
\[
\text{Distancia} = \int_{0}^{3} |v(t)| \, dt = \int_{0}^{3} v(t) \, dt = \boxed{ \frac{204}{25} \, \text{m}}
\]
