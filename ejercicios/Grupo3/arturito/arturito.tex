\chapter*{Problema planteado por el maestro}


Se perfora una esfera de metal de radio R > 6 mm haciéndole un "túnel" cilíndrico de 6 mm de largo que pasa por el centro de la esfera y que deja un sólido con forma de "anillo".
Mostrar que el volumen del sólido no depende de R y que es igual a 36Pi mm^3:

\[
f(3) = \sqrt{R^2 - x^2} \Big|_{x=3} = \sqrt{R^2 - 3^2} = \sqrt{R^2 - 9}
\]

\[
2 \cdot 2\pi \int_{f(3)}^{R} x \sqrt{R^2 - x^2} \, dx
\]

\[
= 4\pi \int_{9}^{0} -\frac{1}{2} \sqrt{u} \, du = -2\pi \int_{9}^{0} \sqrt{u} \, du
\]

\[
\text{Sustituyendo: } 
\quad u = R^2 - x^2, \quad du = -2x \, dx, \quad x \, dx = -\frac{1}{2} du
\]

\[
u(f(3)) = u(\sqrt{R^2 - 9}) = R^2 - (R^2 - 9) = 9
\]

\[
u(R) = R^2 - R^2 = 0
\]

\[
-2\pi \int_{9}^{0} \sqrt{u} \, du = 2\pi \int_{0}^{9} \sqrt{u} \, du
\]

\[
2\pi \int_{0}^{9} u^{1/2} \, du = 2\pi \left[ \frac{2u^{3/2}}{3} \right]_0^9
\]

\[
= 2\pi \left( \frac{2(9)^{3/2}}{3} - \frac{2(0)^{3/2}}{3} \right)
\]

\[
= \frac{4\pi \cdot 9^{3/2}}{3} = \frac{4}{3}\pi (27) = 9 \cdot 4\pi = 36\pi
\]