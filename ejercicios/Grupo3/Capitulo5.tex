\chapter*{Ejercicios capitulo 5 ABD Grupo 3}

\section*{Ejercicio 42}
\textbf{Calcular el área bajo la curva \( y = \frac{1}{x} \) en el intervalo \([1, e^3]\).}

El área bajo la curva está dada por la integral definida:
\[
\int_{1}^{e^3} \frac{1}{x} \, dx
\]

\subsection*{Plantear la integral}
Tenemos:
\[
\int_{1}^{e^3} \frac{1}{x} \, dx
\]

\subsection*{Calcular la integral indefinida}
Sabemos que:
\[
\int \frac{1}{x} \, dx = \ln|x| + C
\]

\subsection*{Evaluar la integral definida}
Sustituimos los límites de integración:
\[
\int_{1}^{e^3} \frac{1}{x} \, dx = \left[ \ln x \right]_{1}^{e^3}
\]

\subsection*{Sustituir los límites}
Evaluamos en los límites:
\[
\left[ \ln x \right]_{1}^{e^3} = \ln(e^3) - \ln(1)
\]

\subsection*{Simplificar}
Sabemos que:
\[
\ln(e^3) = 3 \quad \text{y} \quad \ln(1) = 0
\]
Por lo tanto:
\[
\ln(e^3) - \ln(1) = 3 - 0 = 3
\]

\subsection*{Resultado final}
El área bajo la curva es:
\[
\boxed{3}
\]
\newpage

\section*{Ejercicio 64}
\textbf{Encontrar el valor promedio de \( f(x) = e^x + e^{-x} \) en el intervalo \([ \ln \frac{1}{2}, \ln 2 ]\).}

El valor promedio de una función está dado por:
\[
f_{\text{promedio}} = \frac{1}{b-a} \int_{a}^{b} f(x) \, dx
\]
En este caso, \( f(x) = e^x + e^{-x} \), \( a = \ln \frac{1}{2} \) y \( b = \ln 2 \).

\subsection*{Planteamiento}
Sustituimos en la fórmula:
\[
f_{\text{promedio}} = \frac{1}{\ln 2 - \ln \frac{1}{2}} \int_{\ln \frac{1}{2}}^{\ln 2} (e^x + e^{-x}) \, dx
\]

\subsection*{Simplificar los límites del denominador}
Sabemos que:
\[
\ln 2 - \ln \frac{1}{2} = \ln 2 - \ln 2^{-1} = \ln 2 + \ln 2 = 2 \ln 2
\]
Por lo tanto:
\[
f_{\text{promedio}} = \frac{1}{2 \ln 2} \int_{\ln \frac{1}{2}}^{\ln 2} (e^x + e^{-x}) \, dx
\]

\subsection*{Calcular la integral indefinida}
\[
\int (e^x + e^{-x}) \, dx = \int e^x \, dx + \int e^{-x} \, dx
\]
Resolviendo cada término:
\[
\int e^x \, dx = e^x, \quad \int e^{-x} \, dx = -e^{-x}
\]
Por lo tanto:
\[
\int (e^x + e^{-x}) \, dx = e^x - e^{-x} + C
\]

\subsection*{Evaluar la integral definida}
Sustituimos los límites:
\[
\int_{\ln \frac{1}{2}}^{\ln 2} (e^x + e^{-x}) \, dx = \left[ e^x - e^{-x} \right]_{\ln \frac{1}{2}}^{\ln 2}
\]

\subsection*{Sustituir los límites en la función}
Evaluamos:
\[
\left[ e^x - e^{-x} \right]_{\ln \frac{1}{2}}^{\ln 2} = \left( e^{\ln 2} - e^{-\ln 2} \right) - \left( e^{\ln \frac{1}{2}} - e^{-\ln \frac{1}{2}} \right)
\]
Simplificamos:
\[
e^{\ln 2} = 2, \quad e^{-\ln 2} = \frac{1}{2}, \quad e^{\ln \frac{1}{2}} = \frac{1}{2}, \quad e^{-\ln \frac{1}{2}} = 2
\]
Por lo tanto:
\[
\left( e^{\ln 2} - e^{-\ln 2} \right) - \left( e^{\ln \frac{1}{2}} - e^{-\ln \frac{1}{2}} \right) = \left( 2 - \frac{1}{2} \right) - \left( \frac{1}{2} - 2 \right)
\]
Simplificamos:
\[
\left( 2 - \frac{1}{2} \right) - \left( \frac{1}{2} - 2 \right) = \frac{3}{2} + \frac{3}{2} = 3
\]

\subsection*{Calcular el valor promedio}
Sustituimos en la fórmula del promedio:
\[
f_{\text{promedio}} = \frac{1}{2 \ln 2} \cdot 3 = \frac{3}{2 \ln 2}
\]

\subsection*{Resultado final}
El valor promedio es:
\[
\boxed{\frac{3}{2 \ln 2}}
\]
 \newpage 
 \section*{Ejercicio 76}

La función de velocidad dada es:
\[
v(t) = \frac{2}{5} \sqrt{5t + 1} + \frac{8}{5}.
\]

Queremos calcular el \textbf{desplazamiento} y la \textbf{distancia} recorrida por la partícula en el intervalo \( [0, 3] \).

\subsection*{Desplazamiento}

El desplazamiento se calcula como la integral de la velocidad:
\[
\text{Desplazamiento} = \int_{0}^{3} v(t) \, dt.
\]
Sustituimos \( v(t) \) en la integral:
\[
\int_{0}^{3} v(t) \, dt = \int_{0}^{3} \left( \frac{2}{5} \sqrt{5t + 1} + \frac{8}{5} \right) \, dt.
\]
Separando la integral:
\[
\int_{0}^{3} v(t) \, dt = \frac{2}{5} \int_{0}^{3} \sqrt{5t + 1} \, dt + \frac{8}{5} \int_{0}^{3} 1 \, dt.
\]

\subsubsection*{Primera integral: \( \int_{0}^{3} \sqrt{5t + 1} \, dt \)}

Sea \( u = 5t + 1 \), por lo tanto:
\[
du = 5 \, dt \quad \text{y} \quad dt = \frac{1}{5} \, du.
\]
Cuando \( t = 0 \), \( u = 1 \); y cuando \( t = 3 \), \( u = 16 \). Sustituyendo:
\[
\int_{0}^{3} \sqrt{5t + 1} \, dt = \int_{1}^{16} \sqrt{u} \cdot \frac{1}{5} \, du = \frac{1}{5} \int_{1}^{16} u^{1/2} \, du.
\]
La integral de \( u^{1/2} \) es:
\[
\int u^{1/2} \, du = \frac{2}{3} u^{3/2}.
\]
Entonces:
\[
\frac{1}{5} \int_{1}^{16} u^{1/2} \, du = \frac{1}{5} \left[ \frac{2}{3} u^{3/2} \right]_{1}^{16} = \frac{2}{15} \left[ u^{3/2} \right]_{1}^{16}.
\]
Evaluamos los límites:
\[
\frac{2}{15} \left[ 16^{3/2} - 1^{3/2} \right] = \frac{2}{15} \left[ (16)^{3/2} - 1 \right].
\]
Sabemos que \( 16^{3/2} = (16^{1/2})^3 = 4^3 = 64 \), entonces:
\[
\frac{2}{15} \left[ 64 - 1 \right] = \frac{2}{15} \cdot 63 = \frac{126}{15} = \frac{42}{5}.
\]

\subsubsection*{Segunda integral: \( \int_{0}^{3} 1 \, dt \)}

La integral es directa:
\[
\int_{0}^{3} 1 \, dt = \left[ t \right]_{0}^{3} = 3 - 0 = 3.
\]

\subsubsection*{Combinando ambas integrales}

Sustituyendo los resultados en la expresión original:
\[
\int_{0}^{3} v(t) \, dt = \frac{2}{5} \cdot \frac{42}{5} + \frac{8}{5} \cdot 3 = \frac{84}{25} + \frac{24}{5}.
\]
Simplificamos \( \frac{24}{5} \) a denominador \( 25 \):
\[
\frac{24}{5} = \frac{120}{25}.
\]
Entonces:
\[
\int_{0}^{3} v(t) \, dt = \frac{84}{25} + \frac{120}{25} = \frac{204}{25}.
\]

Por lo tanto, el desplazamiento es:
\[\boxed{
\text{Desplazamiento} = \frac{204}{25} \, \text{m}}
\]

\subsection*{Distancia recorrida}

La distancia recorrida es la misma que el desplazamiento, ya que \( v(t) \geq 0 \) en todo el intervalo \( [0, 3] \). Entonces:
\[
\text{Distancia} = \int_{0}^{3} |v(t)| \, dt = \int_{0}^{3} v(t) \, dt = \boxed{ \frac{204}{25} \, \text{m}}
\]
