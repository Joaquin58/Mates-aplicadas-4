\chapter*{capitulo 7 ejercicio 7}
Un resorte ejerce una fuerza de $0.5N$ cuando se estira $0.25m$ más allá de su lognitud natural. Suponiendo que e aplica la ley de Hooke.
a) ¿Cuánto trabajo se realizó para estirar el resorte hasta esta longitud?
b) ¿Cuánto más allá de su longitud natural se puede estirar el resorte con $25J$ de trabajo?

Sabemos que $W=\displaystyle\int_{a}^{b}F(x)dx$
donde $F=F(x)$ es la fuerza.
$\therefore$
\begin{align*}
	F=F(x)&=k\cdot x\\
	0.5&=k\cdot (0.25)\\
    \implies k&=\frac{0.5}{0.25}=2
\end{align*}
Susutituimos en la integral.
\begin{align*}
    W=\int_{0}^{0.25}2xdx&=2\int_{0}^{0.25}xdx\\
&= 2\frac{x^2}{2}\bigg|_{0}^{0.25}=x^2\bigg|_{0}^{0.25}\\
&=(0.25)^2=(\frac{1}{4})^2=\frac{1}{16}J
\end{align*}

Concluimos que $16J$ es el trabajo realizado.

b) ¿Cuándo pasa que $W=25J$?

Sabemos que el trabajo está descrito como la ecuación $W=x^2$ para nuetro resorte.\\
$\implies$ $x^2=25\iff x=\sqrt{25}=5\;m$ \\
Por lo tanto 5 m es la longitud a la que se estira el resorte al aplicarle un trabajo de $25 J$
